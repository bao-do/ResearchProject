\documentclass[a4paper,10pt]{article}

\usepackage{geometry}
\usepackage{graphicx} % Required for inserting images
\usepackage{amsmath}
\usepackage{amssymb}
\usepackage{amsthm}
\usepackage{mathtools}
\usepackage{bbm}
\usepackage{xcolor}
\usepackage{hyperref}

\theoremstyle{definition} % Style pour les définitions
\newtheorem{definition}{Définition}[section]

\theoremstyle{definition} % Style pour les propositions et lemmes
\newtheorem{proposition}[definition]{Proposition}
\newtheorem{lemma}[definition]{Lemme}

\theoremstyle{definition} % Style pour les remarques
\newtheorem{theorem}[definition]{Théorème}

\theoremstyle{definition} % Style pour les remarques
\newtheorem{remark}[definition]{Remarque}

\newcommand{\E}[1]{\mathbb{E}\left[#1\right]}
\newcommand{\R}{\mathbb{R}}
\newcommand{\N}{\mathbb{N}}
\newcommand{\C}{\mathbb{C}}
\newcommand{\Xc}{\mathcal{X}}
\newcommand{\Ac}{\mathcal{A}}
\newcommand{\dist}{\mathrm{dist}}
\newcommand{\eqdef}{\stackrel{\mathrm{def}}{=}}
\newcommand{\argmin}{\mathop{\mathrm{argmin}}}
\newcommand{\argmax}{\mathop{\mathrm{argmax}}}
\newcommand{\xmap}{x_{\scalebox{0.5}{\textrm{MAP}}}}
\newcommand{\xmmse}{x_{\scalebox{0.5}{\textrm{MMSE}}}}
\newcommand{\xmarginal}{x_{\scalebox{0.5}{\textrm{MARG}}}}
\newcommand{\todo}[1]{\textcolor{red}{\textbf{TODO:} #1}}


\title{Projet de Recherche}
\author{Quoc-Bao DO}
\date{Fevrier 2024}


\begin{document}

\maketitle

\section{Modèle de diffusion}

    En modélisation probabiliste, il est fréquent de devoir générer des échantillons à partir d'une distribution de données dont la forme précise est inconnue ou trop complexe pour permettre un échantillonnage direct. Les modèles de diffusion offrent une solution à ce problème en apprenant une équation différentielle inhomogène dans le temps, qui transforme progressivement des échantillons issus d'une distribution gaussienne simple en échantillons correspondant à la distribution cible, plus complexe.

Considérons une équation différentielle stochastique dépendante du temps (Itô), donnée comme suit :
    \begin{equation}\label{eq:SDE}
         d\phi_t = f_t(\phi_t)dt + g_tdW_t
    \end{equation}
    avec 
    \begin{itemize}
        \item $\phi \in \R^N$
        \item $f_t:\R^N \rightarrow \R^N$ de classe $C^1$, et $f_t$ linéairement croissante, i.e $\exists C > 0$ tel que $\|f_t(\phi)\| \leq C(1+|\ \phi \|)$
        \item $g_t : \R \rightarrow \R$ de classe $C^2$
        \item $W_t$ est un processus de Wiener standard de dimension N, c'est-à-dire tel que 
            \begin{equation}
                \langle W_t^i, W_t^j \rangle =
                    \begin{cases}    
                        t & \textrm{ si } i=j \\
                        0 & \textrm{sinon},
                    \end{cases}
            \end{equation}
            où $W^i$ et $W^j$ sont les $i$-ème et $j$-ème coordonnées de $W_t$ respectivement. 
        % \item $\frac{\partial \pi_t}{\partial t}$ existe $\forall t \geq 0$
        % \item $\pi_0$ décroit rapidement à l'infini, i.e $lim_{\|\phi\| \rightarrow \infty} \| \phi \|^k |\pi_0(\phi)| = 0$
    \end{itemize}
    \begin{remark}
        La condition que $f_t$ soit linéairement croissante rassure l'existence de solution de la SDE.
    \end{remark}

\begin{proposition}[Fokker – Planck equation] 
Sous les hypothèses précédentes, le flux sur les distributions de probabilité $\pi_t$ de $\phi_t$ pour $t \geq 0$ est donné par :
    \begin{equation}
        \frac{\partial\pi_t}{\partial t} = -\nabla \cdot (f_t \pi_t) + \frac{1}{2}\nabla^2(g_t^2\pi_t)
    \end{equation}
\end{proposition}
\begin{proof}
Commençons par un rappel sur la formule d'Itô . 
\begin{theorem}[Formule d'Itô \cite{Oksendal2003}]
Soit \( X(t) \) un processus d’Itô de dimension \( n \) vérifiant  
\[
dX(t) = u(t, X(t)) dt + v(t, X(t)) dB(t),
\]
où :
\begin{itemize}
    \item \( X(t) \in \mathbb{R}^n \) est le processus d’état,
    \item \( u(t, X(t)) \in \mathbb{R}^n \) est le terme de dérive,
    \item \( v(t, X(t)) \in \mathbb{R}^{n \times m} \) est la matrice de diffusion,
    \item \( B(t) \in \mathbb{R}^m \) est un mouvement brownien standard de dimension \( m \).
\end{itemize}

Soit \( g: [0, \infty) \times \mathbb{R}^n \to \mathbb{R}^p \) une fonction deux fois continûment différentiable.  
On définit le processus transformé :
\[
Y(t) = g(t, X(t)),
\]
où \( Y(t) \in \mathbb{R}^p \).  
Alors, \( Y(t) \) satisfait l’équation d’Itô :
\[
dY_k = \frac{\partial g_k}{\partial t} dt 
+ \sum_{i=1}^{n} \frac{\partial g_k}{\partial x_i} u_i dt 
+ \sum_{i=1}^{n} \sum_{j=1}^{m} \frac{\partial g_k}{\partial x_i} v_{ij} dB_j
+ \frac{1}{2} \sum_{i,j=1}^{n} \sum_{r=1}^{m} v_{ir} v_{jr} \frac{\partial^2 g_k}{\partial x_i \partial x_j} dt.
\]
\end{theorem}
\vspace{3em}



Soit une fonction test $F: \mathbb{R}^N \rightarrow \mathbb{R}$, de classe $C^{\infty}$, à support compact. En utilisant la formule d'Ito, on obtient
%     \[dF(\phi_t) = \sum_{i=1}^N \frac{\partial F(\phi_t)}{\partial x_i} d\phi_t^i + \frac{1}{2} \sum_{i=1}^N \sum_{j=1}^N \frac{\partial^2 F(\phi_t)}{\partial x_i^2}d\langle \phi^i,\phi^j\rangle_t\]
% En utilisant l'équation \eqref{eq:SDE} et en notant que 
%  on obtient:
\begin{align*}
    dF(\phi_t) &= \sum_{i=1}^N \frac{\partial F(\phi_t)}{\partial x_i} (f^i_t(\phi_t)dt + g_tdW^i_t) + \frac{1}{2} \sum_{i=1}^N  \frac{\partial^2 F(\phi_t)}{\partial x_i^2} g_t^2 dt \\
    &= \left(\sum_{i=1}^N \frac{\partial F(\phi_t)}{\partial x_i} f^i_t(\phi_t) + \frac{1}{2} \sum_{i=1}^N  \frac{\partial^2 F(\phi_t)}{\partial x_i^2} g_t^2\right)dt + \sum_{i=1}^N \frac{\partial F(\phi_t)}{\partial x_i}  g_t^2dW^i_t
\end{align*}

Le terme $dW_t^i$ disparaît en prenant l'espérance  (car $\mathbb{E}[dW_t^i] = 0$ ), donc :

\[\E{dF(\phi_t)} = \E{\left(\sum_{i=1}^N \frac{\partial F(\phi_t)}{\partial x_i} f^i_t(\phi_t) + \frac{1}{2} \sum_{i=1}^N \frac{\partial^2 F(\phi_t)}{\partial x_i^2} g_t^2\right)dt}\]

Ou encore, en utilisant la linéarité de l'opérateur espérance, on peut sortir la dérivée, l'équation ci-dessus s'écrit :

\[ d\E{F(\phi_t)} = \E{\sum_{i=1}^N \frac{\partial F(\phi_t)}{\partial x_i} f^i_t(\phi_t) + \frac{1}{2} \sum_{i=1}^N \frac{\partial^2 F(\phi_t)}{\partial x_i^2} g_t^2}dt\]

Ainsi,
\begin{align}
    \frac{d\E{F(\phi_t)}}{dt} &= \E{\sum_{i=1}^N \frac{\partial F(\phi_t)}{\partial x_i} f^i_t(\phi_t) + \frac{1}{2} \sum_{i=1}^N  \frac{\partial^2 F(\phi_t)}{\partial x_i^2}g_t^2} \\
    &=\E{\sum_{i=1}^N \frac{\partial F(\phi_t)}{\partial x_i} f^i_t(\phi_t)} + \E{\frac{1}{2} \sum_{i=1}^N  \frac{\partial^2 F(\phi_t)}{\partial x_i^2} g_t^2}\\
    &=\E{ \nabla\ F(\phi_t) \cdot f_t(\phi_t)} + \frac{1}{2}\E{\nabla^2 F(\phi_t) g_t^2}
\end{align}

$F$ est continue, donc mesurable, supposons que $\phi \rightarrow \frac{\partial\pi_t(\phi)}{\partial t}$ existe, $\phi \rightarrow F(\phi)\frac{\partial\pi_t(\phi)}{\partial t}$ est intégrable en $\R^N$ car F est à support compact. D'après la règle de Leibniz, le premier terme à droite s'écrit :

\[\frac{d\mathbb{E}[F(\phi_t)]}{dt} = \frac{d (\int_{\mathbb{R^N}} F(\phi) \pi_t(\phi) d\phi)} {dt} = \int_{\mathbb{R^N}}F(\phi)\frac{\partial\pi_t(\phi)}{\partial t} d\phi\]

Nous allons rappeler la première identité de Green (Green's first identity) qui est fort utile dans la suite.


\begin{theorem}[Première identité de Green]
Soit $\Omega \subset \R^n$, on se donne une fonction $u : \Omega \rightarrow \R$, $u \in C^2$ et une autre fonction $v : \Omega \rightarrow \R$, $v \in C^1$, alors :
\begin{equation}\label{eq:GreenId}
\int_\Omega (v(x)\nabla^2 u(x) + \nabla u(x) \cdot \nabla v(x)) dx = \int_{\partial\Omega} v \nabla u \cdot n dS
\end{equation}
Avec n le vecteur normal unitaire à la frontière $\partial\Omega$ et $dS$ la mesure de frontière sur $\partial\Omega$.
\end{theorem}
\vspace{3em}

Supposons qu'il existe une fonction $v_t$ telle que $\nabla_{\phi} v_t = f_t \pi_t$.

On utilise la première identité de Green sur le premier terme à gauche en remarquant que l'intégrale sur le bord s'annule car F est à support compact, on obtient :
\begin{align*}
    \E{\nabla\ F(\phi_t)\cdot f_t(\phi_t)}  &=\int_{\mathbb{R^N}} \nabla\ F(\phi)\cdot \big(f_t(\phi) \pi_t(\phi) \bigr)d\phi \\
    &= - \int_{\mathbb{R}^N}F(\phi) (\nabla \cdot f_t(\phi)\pi_t(\phi))d\phi
\end{align*}

Pareil, en appliquant l'intégration par parties deux fois sur le deuxième terme à droite, on obtient :
\begin{align*}
    \E{\nabla^2 F(\phi_t) g_t^2} &= \int_{\mathbb{R}^N } \nabla^2 F(\phi) g_t^2 \pi_t(\phi)d\phi\\
    &= -\int_{\R^N}\nabla F(\phi) \cdot\nabla(g_t^2\pi_t(\phi)) d\phi\\
    &= \int_{\mathbb{R}^N } F(\phi) \nabla^2(g_t^2 \pi_t(\phi)) d\phi
\end{align*}

Injectons les résultats précédents dans l'équation (4) :
\[\int_{\mathbb{R^N}}F(\phi)\frac{\partial\pi_t(\phi)}{\partial t} d\phi = - \int_{\mathbb{R}^N}F(\phi) (\nabla \cdot f_t(\phi)\pi_t(\phi))d\phi + \frac{1}{2}\int_{\mathbb{R}^N } F(\phi) \nabla^2(g_t^2 \pi_t(\phi)) d\phi\]

Cette relation est vraie pour toute fonction de test $F \in C_c^{\infty}$ (i.e ensemble des fonctions infiniment dérivables à support compact), la théorie de distribution nous dit : 
\[\frac{\partial\pi_t(\phi)}{\partial t} = -\nabla \cdot (f_t(\phi)\pi_t(\phi)) + \frac{1}{2}\nabla^2(g_t^2 \pi_t(\phi)) \quad \] 
\end{proof}


Le processus avant est généralement construit de sorte que lorsque $t \rightarrow \infty$ (ou $t \rightarrow T$ pour un certain temps fini $T$), la distribution $\pi_t$ converge vers une distribution connue et bien définie $\pi_{\infty}$, souvent choisie comme une gaussienne à variance finie.

Dans le contexte des modèles de diffusion, et plus précisément des modèles implicites de diffusion pour le débruitage (DDIMs), on cherche un champ de vecteurs déterministe et dépendant du temps $v_t(\phi)$ qui reproduit la même transformation des distributions de probabilité que l'équation stochastique précédente. Cette reformulation permet d'inverser le processus de diffusion de manière déterministe :
\begin{itemize}
    \item On commence par échantillonner $\phi_T \sim \pi_t$
    \item Puis on fait évoluer l'échantillon en arrière dans temps, de $t=T$ à $t = 0$, en résolvant l'EDO suivante :
    \begin{equation}
        \frac{d\phi_t}{dt} = v_t(\phi_t)
    \end{equation}
\end{itemize}

D'après l'équation de Fokker-Planck, l'évolution de la distribution $\phi_t$ est donnée par l'équation de transport suivante :
\[\frac{d\pi_t(\phi)}{dt} =-\nabla \cdot [v_t(\phi)\pi_t(\phi)]\]

Notre but est d'identifier la fonction $v_t$ déterministe et dépendante en temps de sorte que l'équation au dessus produise la même évolution que l'équation (2) (flow-matching en anglais). Pour ce faire, on récrit (2) comme suit :
\[\frac{d\pi_t(\phi)}{dt} =\nabla \cdot([f_t(\phi)-\frac{1}{2}g_t^2\nabla log\pi_t(\phi)]\pi_t(\phi))\]

Ainsi,
\[v_t(\phi) =f_t(\phi)-\frac{1}{2}g_t^2\nabla log\pi_t(\phi) \]
On note $s_t(\phi) = \nabla log\pi_t(\phi)$ et on appelle $s_t$ la fonction de score (score function en anglais).\\

Le choix le plus courant de processus direct est un processus Ornstein-Uhlenbeck inhomogène de la forme suivante :
\[d\phi_t = -\gamma_t\phi_tdt + \sqrt{2\gamma_t}dW_t\]
Ainsi, le fluide de probabilité est donné par :
\[v_t(\phi) = -\gamma_t(\phi+\nabla log \pi_t(\phi)) = -\gamma_t(\phi+s_t(\phi))\]


\bibliographystyle{plain}
\bibliography{bibliography} 
\end{document}