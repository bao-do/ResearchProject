\documentclass{article}
\newcommand{\vmargin}{2cm}
\newcommand{\hmargin}{2.5cm}
\usepackage[a4paper,top=\vmargin,bottom=\vmargin,left=\hmargin,right=\hmargin]{geometry}
\usepackage{graphicx} % Required for inserting images
\usepackage{amsmath}
\usepackage{amssymb}
\usepackage{mathtools}
\setlength{\parindent}{0pt}
\usepackage{bbm}

\usepackage{amsthm}
\theoremstyle{definition} % Style pour les définitions
\newtheorem{definition}{Définition}[section]

\theoremstyle{definition} % Style pour les propositions et lemmes
\newtheorem{proposition}[definition]{Proposition}
\newtheorem{lemma}[definition]{Lemme}

\theoremstyle{definition} % Style pour les remarques
\newtheorem{theorem}[definition]{Théorème}

\theoremstyle{definition} % Style pour les remarques
\newtheorem{remark}[definition]{Remarque}


\title{Projet de Recherche}
\author{Quoc-Bao DO}
\date{Fevrier 2024}


\begin{document}

\maketitle

\section{Modèle de diffusion}

    En modélisation probabiliste, il est fréquent de devoir générer des échantillons à partir d'une distribution de données dont la forme précise est inconnue ou trop complexe pour permettre un échantillonnage direct. Les modèles de diffusion offrent une solution à ce problème en apprenant une équation différentielle inhomogène dans le temps, qui transforme progressivement des échantillons issus d'une distribution gaussienne simple en échantillons correspondant à la distribution cible, plus complexe.
\begin{proposition}
    \textbf{(Fokker – Planck equation)} Considérons une équation différentielle stochastique dépendante du temps (Ito), donnée comme suit :
    \begin{equation}
         d\phi_t = f_t(\phi_t)dt + g_tdW_t
    \end{equation}
    avec $\phi \in \mathbb{R}^N$, $f_t:\mathbb{R_+} \rightarrow R^N$ de classe $C^1$, $g_t : \mathbb{R} \rightarrow \mathbb{R}$ de classe $C^2$ , et $W_t$ est un processus Wiener standard de dimension N.\\
    Alors le flux sur les distributions de probabilité $\pi_t$ de $\phi_t$ pour $t \geq 0$ est donné par :
    \begin{equation}
        \frac{\partial\pi_t(\phi)}{\partial t} = -\nabla \cdot (f_t(\phi) \pi_t(\phi)) + \frac{1}{2}\nabla^2(g_t^2\pi_t(\phi))
    \end{equation}
\end{proposition}

\textbf{Démonstration:}\\

Soit une fonction test $F: \mathbb{R}^N \rightarrow \mathbb{R}$, de classe $C^2$, alors d'après le lemme d’Itô :

    \[dF(\phi_t) = \sum_{i=1}^N \frac{\partial F(\phi_t)}{\partial x_i} d\phi_t^i + \frac{1}{2} \sum_{i=1}^N \sum_{j=1}^N \frac{\partial^2 F(\phi_t)}{\partial x_i^2}d\langle \phi^i,\phi^j\rangle_t\]
D'après l'équation (1) et notons que $\langle W_t^i, W_t^j \rangle_t$ si $i = j$ et $0$ sinon, avec $W^i$ et $W^j$ sont les $i$-èmes et $j$-èmes coordonnées de $W_t$ respectivement, ainsi,
\begin{align*}
    dF(\phi_t) &= \sum_{i=1}^N \frac{\partial F(\phi_t)}{\partial x_i} d\phi_t^i + \frac{1}{2} \sum_{i=1}^N  \frac{\partial^2 F(\phi_t)}{\partial x_i^2} g_t^2 dt \\
    &= \sum_{i=1}^N \frac{\partial F(\phi_t)}{\partial x_i} (f^i_t(\phi_t)dt + g_tdW^i_t) + \frac{1}{2} \sum_{i=1}^N  \frac{\partial^2 F(\phi_t)}{\partial x_i^2} g_t^2 dt \\
    &= (\sum_{i=1}^N \frac{\partial F(\phi_t)}{\partial x_i} f^i_t(\phi_t) + \frac{1}{2} \sum_{i=1}^N  \frac{\partial^2 F(\phi_t)}{\partial x_i^2} g_t^2)dt + \sum_{i=1}^N \frac{\partial F(\phi_t)}{\partial x_i}  g_tdW^i_t
\end{align*}

Le terme $dW_t^i$ disparait en prenant l'esperance  (car $\mathbb{E}[dW_t^i] = 0$ ), donc :

\[\mathbb{E}[dF(\phi_t)] = \mathbb{E}[(\sum_{i=1}^N \frac{\partial F(\phi_t)}{\partial x_i} f^i_t(\phi_t) + \frac{1}{2} \sum_{i=1}^N \frac{\partial^2 F(\phi_t)}{\partial x_i^2} g_t^2)] dt\]

Ainsi,

\begin{align}
    \frac{d\mathbb{E}[F(\phi_t)]}{dt} &= \mathbb{E}[\sum_{i=1}^N \frac{\partial F(\phi_t)}{\partial x_i} f^i_t(\phi_t) + \frac{1}{2} \sum_{i=1}^N  \frac{\partial^2 F(\phi_t)}{\partial x_i^2}g_t^2 \\
    &=\mathbb{E}[\sum_{i=1}^N \frac{\partial F(\phi_t)}{\partial x_i} f^i_t(\phi_t)] + \mathbb{E}[\frac{1}{2} \sum_{i=1}^N  \frac{\partial^2 F(\phi_t)}{\partial x_i^2} g_t^2]
\end{align}

$F$ est continue, donc mesurable, supposons que $\frac{\partial\pi_t(\phi)}{\partial t}$ est intégrable, alors le terme à gauche s'écrit :

\[\frac{d\mathbb{E}[F(\phi_t)]}{dt} = \frac{d (\int_{\mathbb{R^N}} F(\phi) \pi_t(\phi) d\phi)} {dt} = \int_{\mathbb{R^N}}F(\phi)\frac{\partial\pi_t(\phi)}{\partial t} d\phi\]

Comme $\pi_t$ s'annule rapidement à l'infini, l’application du théorème de Green sur le premier terme à droite donne:
\begin{align*}
    \mathbb{E}[\sum_{i=1}^N \frac{\partial F(\phi_t)}{\partial x_i} f^i_t(\phi_t)] dt &= \int_{\mathbb{R^N}}\sum_{i=1}^N \frac{\partial F(\phi)}{\partial x_i} f^i_t(\phi) \pi_t(\phi)d\phi\\
    &=\sum_{i=1}^N \int_{\mathbb{R^N}} \frac{\partial F(\phi)}{\partial x_i} f^i_t(\phi) \pi_t(\phi)d\phi \\
    &= \sum_{i=1}^N(- \int_{\mathbb{R^N}} F(\phi)\frac{\partial (f^i_t(\phi) \pi_t(\phi))}{\partial x_i} d\phi)\\
    &= - \int_{\mathbb{R}^N}F(\phi) (\nabla \cdot f_t(\phi)\pi_t(\phi))d\phi
\end{align*}

Pareil, en appliquant l'intégration par parties deux fois sur le deuxième terme à droite, on obtient :
\begin{align*}
    \mathbb{E}[\frac{1}{2} \sum_{i=1}^N\frac{\partial^2 F(\phi_t)}{\partial x_i^2} g_t^2] &= \sum_{i=1}^N \int_{\mathbb{R}^N } \frac{\partial^2 F(\phi)}{\partial x_i^2} g_t^2 \pi_t(\phi)d\phi\\
    &=\sum_{i=1}^N \int_{\mathbb{R}^N } F(\phi) \frac{\partial^2}{\partial x_i^2} (g_t^2 \pi_t(\phi)) d\phi \\
    &= \int_{\mathbb{R}^N } F(\phi) \nabla^2(g_t^2 \pi_t(\phi)) d\phi
\end{align*}

Injectons les résultats précédents dans l'équation (4) :
\[\int_{\mathbb{R^N}}F(\phi)\frac{\partial\pi_t(\phi)}{\partial t} d\phi = - \int_{\mathbb{R}^N}F(\phi) (\nabla \cdot f_t(\phi)\pi_t(\phi))d\phi + \int_{\mathbb{R}^N } F(\phi) \nabla^2(g_t^2 \pi_t(\phi)) d\phi\]

Cette relation est vraie pour toute fonction test F, d'où le résultat final : 
\[\frac{\partial\pi_t(\phi)}{\partial t} = -\nabla \cdot (f_t(\phi)\pi_t(\phi)) + \nabla^2(g_t^2 \pi_t(\phi)) \] 

Le processus avant est généralement construit de sorte que lorsque $t \rightarrow \infty$ (ou $t \rightarrow T$ pour un certain temps fini $T$), la distribution $\pi_t$ converge vers une distribution connue et bien définie $\pi_{\infty}$, souvent choisie comme une gaussienne à variance finie.

Dans le contexte des modèles de diffusion, et plus précisément des modèles implicites de diffusion pour le débruitage (DDIMs), on cherche un champ de vecteurs déterministe et dépendant du temps $v_t(\phi)$ qui reproduit la même transformation des distributions de probabilité que l'équation stochastique précédente. Cette reformulation permet d'inverser le processus de diffusion de manière déterministe :
\begin{itemize}
    \item On commence par échantillonner $\phi_T \sim \pi_t$
    \item Puis on fait évoluer l'échantillon en arrière dans temps, de $t=T$ à $t = 0$, en résolvant l'EDO suivante :
    \begin{equation}
        \frac{d\phi_t}{dt} = v_t(\phi_t)
    \end{equation}
\end{itemize}

D'après l'équation de Fokker-Planck, l'évolution de la distribution $\phi_t$ est donnée par l'équation de transport suivante :
\[\frac{d\pi_t(\phi)}{dt} =-\nabla \cdot [v_t(\phi)\pi_t(\phi)]\]

Notre but est d'identifier la fonction $v_t$ déterministe et dépendante en temps de sorte que l'équation au dessus produise la même évolution que l'équation (2) (flow-matching en anglais). Pour ce faire, on récrit (2) comme suit :
\[\frac{d\pi_t(\phi)}{dt} =\nabla \cdot([f_t(\phi)-\frac{1}{2}g_t^2\nabla log\pi_t(\phi)]\pi_t(\phi))\]

Ainsi,
\[v_t(\phi) =f_t(\phi)-\frac{1}{2}g_t^2\nabla log\pi_t(\phi) \]
On note $s_t(\phi) = \nabla log\pi_t(\phi)$ et on appelle $s_t$ la fonction de score (score function en anglais).\\

Le choix le plus courant de processus direct est un processus Ornstein-Uhlenbeck inhomogène de la forme suivante :
\[d\phi_t = -\gamma_t\phi_tdt + \sqrt{2\gamma_t}dW_t\]
Ainsi, le fluide de probabilité est donné par :
\[v_t(\phi) = -\gamma_t(\phi+\nabla log \pi_t(\phi)) = -\gamma_t(\phi+s_t(\phi))\]

\end{document}